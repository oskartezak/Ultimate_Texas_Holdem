\documentclass[a4paper,12pt]{article}

% ----------------------------------
% Paketi za slovenščino in znanstveno pisanje
% ----------------------------------
\usepackage[slovene]{babel}
\usepackage[utf8]{inputenc}
\usepackage[T1]{fontenc}

% ----------------------------------
% Oblika strani
% ----------------------------------
\usepackage{geometry}
\geometry{margin=2.5cm}

% ----------------------------------
% Matematični simboli in formule
% ----------------------------------
\usepackage{amsmath, amssymb}

% ----------------------------------
% Slike in tabele
% ----------------------------------
\usepackage{graphicx}
\usepackage{caption}
\usepackage{subcaption}
\usepackage{booktabs} % lepe tabele

% ----------------------------------
% Koda (če jo želiš prikazati)
% ----------------------------------
\usepackage{listings}
\usepackage{xcolor}
\lstset{
    basicstyle=\ttfamily\small,
    backgroundcolor=\color{gray!10},
    frame=single,
    breaklines=true
}

% ----------------------------------
% Naslov
% ----------------------------------
\title{Poročilo projekta pri predmetu Izbrane teme iz analize podatkov}
\author{Neo Mistral, Oskar Težak \\ Fakulteta za matematiko in fiziko, Univerza v Ljubljani}
\date{\today}

\begin{document}

\maketitle

\section{Uvod}

Cilj projekta je razviti model z uporabo \textbf{nevronskih mrež}, 
ki bi znal poiskati optimalno strategijo za igranje igre \textit{Texas Hold'em} pokra.

\section{Generiranje in oblike podatkov}

Podatke sva generirala naključno. Igra temelji na 52 kartah, ki sva jih označila s številkami od 1 do 52. 
Vsak igralec, vključno z delilcem, prejme dve karti, na mizi pa je skupaj pet kart: najprej trije t.~i.~\textit{flop} 
in nato še dve karti, ki skupaj s prvimi tremi tvorita \textit{river}.

Podatke sva vedno generirala za celotno igro, da sva lahko izračunala njen izid. V model pa sva nato vključila le tiste karte, 
ki so bile na voljo v določenem krogu igre. Te karte so predstavljale vhodne spremenljivke:
\begin{itemize}
    \item v prvem krogu je model videl samo dve karti igralca (ostala mesta so bila nastavljena na 0),
    \item v drugem krogu pet kart,
    \item v tretjem krogu vseh sedem kart.
\end{itemize}

Model nikoli ni imel vpogleda v delilčeve karte. V primeru več igralcev pa je model upošteval tudi njihove karte.

Drugi del podatkov so predstavljale \textbf{ciljne vrednosti}. Preizkusila sva več načinov njihovega določanja:
\begin{enumerate}
    \item \textbf{Povprečenje iger} -- za vsak par igralčevih kart sva generirala več možnih kombinacij preostalih kart in izračunala povprečen izid. 
    Ker pa je igra zasnovana tako, da je to povprečje za igralca vedno negativno, se ta pristop ni izkazal za primernega.
    \item \textbf{Binaren izid (zmaga/poraz)} -- kot ciljno spremenljivko sva uporabila enostaven pokazatelj zmage ali poraza (0 ali 1).
    \item \textbf{Izkupiček posamezne igre} -- kot ciljno spremenljivko sva uporabila dejanski izid posamezne igre, brez povprečenja.
\end{enumerate}

Število podatkov

\section{Metodologija}
\begin{itemize}
    \item Opis uporabljenih modelov strojnega učenja
    \item Parametri in nastavitve
    \item Postopek treniranja in validacije
\end{itemize}

\section{Rezultati}
\begin{itemize}
    \item Natančnost, F1, MSE, ipd.
    \item Grafi in tabele rezultatov
\end{itemize}

\section{Zaključek}
Povzetek glavnih ugotovitev in predlogi za nadaljnje delo.

\end{document}

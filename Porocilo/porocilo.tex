\documentclass[a4paper,12pt]{article}

% ----------------------------------
% Paketi za slovenščino in znanstveno pisanje
% ----------------------------------
\usepackage[slovene]{babel}
\usepackage[utf8]{inputenc}
\usepackage[T1]{fontenc}

% ----------------------------------
% Oblika strani
% ----------------------------------
\usepackage{geometry}
\geometry{margin=2.5cm}

% ----------------------------------
% Matematični simboli in formule
% ----------------------------------
\usepackage{amsmath, amssymb}

% ----------------------------------
% Slike in tabele
% ----------------------------------
\usepackage{graphicx}
\usepackage{caption}
\usepackage{subcaption}
\usepackage{booktabs} % lepe tabele

% ----------------------------------
% Koda (če jo želiš prikazati)
% ----------------------------------
\usepackage{listings}
\usepackage{xcolor}
\lstset{
    basicstyle=\ttfamily\small,
    backgroundcolor=\color{gray!10},
    frame=single,
    breaklines=true
}

% ----------------------------------
% Naslov
% ----------------------------------
\title{Poročilo projekta pri predmetu Izbrane teme iz analize podatkov}
\author{Neo Mistral, Oskar Težak \\ Fakulteta za matematiko in fiziko, Univerza v Ljubljani}
\date{\today}

\begin{document}

\maketitle

\section{Uvod}

Cilj projekta je razviti model z uporabo \textbf{nevronskih mrež}, 
ki bi znal poiskati optimalno strategijo za igranje igre \textit{Texas Hold'em} pokra.
\section{Opis igre Ultimate Texas Hold'em}

\textbf{Ultimate Texas Hold'em} je različica pokra, ki se igra proti delilcu, 
ne pa proti drugim igralcem. Cilj igre je premagati delilca z boljšo 
poker kombinacijo iz dveh lastnih in petih skupnih kart.

\subsection*{Potek igre}
\begin{enumerate}
    \item \textbf{Začetna stava} -- igralec postavi enaka vložka na polji \emph{Ante} in \emph{Blind}. 
    Po želji lahko postavi tudi stransko stavo \emph{Trips}, ki se izplača na osnovi končne kombinacije.
    
    \item \textbf{Deljenje kart} -- igralec in delilec prejmeta po dve skriti karti. 
    Igralec lahko zdaj izbere, da stavi dodatno stavo \emph{Play} v višini $3\times$ ali $4\times$ 
    začetne stave, ali pa počaka.
    
    \item \textbf{Flop} -- razkrijejo se prve tri skupne karte. Če igralec prej ni stavil, 
    lahko zdaj postavi stavo \emph{Play} v višini $2\times$ začetne stave, ali pa še vedno počaka.
    
    \item \textbf{Turn in River} -- razkrijeta se še četrta in peta skupna karta. 
    Če igralec do zdaj ni stavil, mora zdaj postaviti stavo \emph{Play} v višini $1\times$ začetne stave.
    
    \item \textbf{Razkritje} -- delilec razkrije svoji dve karti. 
    Za kvalifikacijo mora imeti vsaj par. Če se delilec ne kvalificira, se stava \emph{Ante} vrne igralcu, 
    stave \emph{Blind} in \emph{Play} pa se obravnavajo glede na primerjavo kombinacij.
    
    \item \textbf{Izplačilo} -- 
    \begin{itemize}
        \item Stava \emph{Play} se vedno izplača $1:1$, če je igralčeva kombinacija boljša.
        \item Stava \emph{Ante} se izplača $1:1$, če se delilec kvalificira in igralec zmaga.
        \item Stava \emph{Blind} se izplača po posebni lestvici, odvisno od moči igralčeve kombinacije (npr. \emph{Straight} $1:1$, \emph{Flush} $3:2$, \emph{Full House} $3:1$, ipd.).
        \item Stranska stava \emph{Trips} se izplača posebej glede na moč kombinacije, neodvisno od izida proti delilcu.
    \end{itemize}
\end{enumerate}

\section{Generiranje in oblike podatkov}

Podatke sva generirala naključno. Igra temelji na 52 kartah, ki sva jih označila s številkami od 1 do 52. 
Vsak igralec, vključno z delilcem, prejme dve karti, na mizi pa je skupaj pet kart: najprej trije t.~i.~\textit{flop} 
in nato še dve karti, ki skupaj s prvimi tremi tvorita \textit{river}.

Podatke sva vedno generirala za celotno igro, da sva lahko izračunala njen izid. V model pa sva nato vključila le tiste karte, 
ki so bile na voljo v določenem krogu igre. Te karte so predstavljale vhodne spremenljivke:
\begin{itemize}
    \item v prvem krogu je model videl samo dve karti igralca (ostala mesta so bila nastavljena na 0),
    \item v drugem krogu pet kart,
    \item v tretjem krogu vseh sedem kart.
\end{itemize}

Model nikoli ni imel vpogleda v delilčeve karte. V primeru več igralcev pa je model upošteval tudi njihove karte.

Drugi del podatkov so predstavljale \textbf{ciljne vrednosti}. Preizkusila sva več načinov njihovega določanja:
\begin{enumerate}
    \item \textbf{Povprečenje iger} -- za vsak par igralčevih kart sva generirala več možnih kombinacij preostalih kart in izračunala povprečen izid na 
    za krog igre. 
    \item \textbf{Binaren izid (zmaga/poraz)} -- kot ciljno spremenljivko sva uporabila enostaven pokazatelj zmage ali poraza (0 ali 1). 
    Tudi te vrednosti sva generirala na enak način kot pri povprečenju, le da sva namesto povprečnega izida upoštevala zmago ali poraz.
    \item \textbf{Izkupiček posamezne igre} -- kot ciljno spremenljivko sva uporabila dejanski izid posamezne igre, brez povprečenja. 
    Tudi ti podatki so bili generirani enako kot pri prvem pristopu, pri čemer sva ohranila rezultat posamezne igre.
\end{enumerate}

\paragraph{Velikost podatkovne zbirke.}
Za igralca je velikost podatkovne zbirke znašala 1326 primerov (vseh možnih začetnih kombinacij igralčevih kart). 
Za preostale karte sva uporabila večkratnike te številke: prvi večkratnik $15$ (za "flop"), drugi $15$ (za "river"), 
in pri zadnjem večkratnik $20$ (za delilca). 

\section{Metodologija}

Pri večini poskusov sva uporabila \textbf{linearne} oziroma 
\textbf{konvolucijske nevronske mreže}. 
Za določanje stavnih pragov sva poleg različnih optimizacijskih pristopov 
uporabila tudi \textbf{lastne ocene}, s katerimi sva dodatno prilagodila 
strategijo in izboljšala delovanje modela.

\subsection{Poskus 1.}

\section{Rezultati}
\begin{itemize}
    \item Natančnost, F1, MSE, ipd.
    \item Grafi in tabele rezultatov
\end{itemize}

\section{Zaključek}
Povzetek glavnih ugotovitev in predlogi za nadaljnje delo.

\end{document}
